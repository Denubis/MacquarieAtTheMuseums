\environment envArts
\starttext

\setupwhitespace[medium]
\setupTABLE[rulethickness=0pt]
\setupTABLE[r][odd][background=color, backgroundcolor=lightgray]
\setupTABLE[style={\tfx}]
\setuphead[section][number=off, style={\tfc}, align={hz, hanging}]
\setuphead[subsection][number=off, style={\tfa}]
\setupitemgroup[itemize][packed]
\setupalign[hz, hanging]



\title{Macquarie at the Museums: \crlf Saturday Conference Program}
\setuppagenumbering[alternative=doublesided, location={footer, right}, state=stop]
\externalfigure[../images/e_10_2010_Khnumhotep_II_D01_DSC_0609(1).jpg][width=\textwidth]


https://mqatthemuseums.com

Location: Maritime Museum

Participation in the Saturday conference is New South Wales Educational Standards Authority (NESA) accredited.



\startitemize[packed]
\item Option 1 - Friday and Saturday night: Code MQMUS1
\item Option 2 - Saturday night: Code MQMUS2
\item Option 3 - Friday night: Code MQMUS3
\stopitemize

NB: Any surplus will go towards developing the Schools Resources Site


\page


\subsection{Session 1}
\setuppagenumbering[alternative=doublesided, location={footer, right}, state=start]
{\bf Chair: A/Prof Tom Hillard, Department of Ancient History}

\blank[1* line]

\bTABLE
\bTR \bTD 8am  \eTD \bTD Tea/coffee  \eTD \bTD \eTD \eTR
\bTR \bTD 8.30 - 8.45  \eTD \bTD A/Prof Tom Hillard  \eTD \bTD Conference Announcements \eTD \eTR
\bTR \bTD 8.45 - 9  \eTD \bTD Vice-Chancellor Bruce S. Dowton  \eTD \bTD Welcome Presentation  \eTD \eTR
\bTR \bTD 9 -- 10  \eTD \bTD Prof. Ray Laurence, keynote address  \eTD \bTD Children in Pompeii -- the invisible majority? \eTD \eTR
\bTR \bTD 10 -- 10.10  \eTD \bTD Ms Philippa Medcalf \eTD \bTD School visits to the Macquarie Museum of Ancient Cultures \eTD \eTR
\bTR \bTD 10.10 -- 10.20  \eTD \bTD Dr Denis Mootz  \eTD \bTD Resources for Schools website \eTD \eTR
\bTR \bTD 10.20 -- 10.30  \eTD \bTD Mr Jeff Fletcher  \eTD \bTD About the Maritime Museum \eTD \eTR
\bTR \bTD 10.30 -- 11  \eTD \bTD Morning tea  \eTD \bTD \eTD \eTR
\eTABLE

\subsection{Session 2}



\startitemize[packed]
\item Chair, Ancient History Stream: Ms Kate Cameron, HTA
\item Chair, Study of Religion Stream: Ms Louise Zavone, ASR
\stopitemize
\blank[1* line]
\bTABLE
\bTR \bTD 11 -- 12.30  \eTD \bTD Stream 1  \eTD \bTD {\sc Ancient History}  \eTD \bTD Stream 2         \eTD \bTD {\sc Study of Religion} \eTD \eTR
\bTR \bTD 11 -- 11.30  \eTD \bTD Dr Kyle Keimer  \eTD \bTD The Early Israelite Monarchy in Text and Archaeology  \eTD \bTD Dr Chris Forbes  \eTD \bTD Historiography of Acts of the Apostles \eTD \eTR
\bTR \bTD 11.30 -- 12  \eTD \bTD Dr Gil Davis  \eTD \bTD Re-dating the Athenian Empire  \eTD \bTD Dr Louise Pryke  \eTD \bTD The Book of Proverbs: Wisdom, Women, Worldliness \eTD \eTR
\bTR \bTD 12 -- 12:30  \eTD \bTD A/Prof Tom Hillard \& Dr Lea Beness  \eTD \bTD The Late Roman Republic in 2017: Recent Developments  \eTD \bTD Ms Katherine Jacka  \eTD \bTD Persian Polymath: Ibn Sīna and the Development of Islamic Philosophy and Science \eTD \eTR
\bTR \bTD 12.30 -- 1.30  \eTD \bTD Lunch \eTD \eTR
\eTABLE

\subsection{Session 3}



\startitemize[packed]
\item Chair: Mr Darren Tayler, NESA.  Inspector HSIE, Curriculum and Assessment Standards
\stopitemize

\blank[1* line]
\bTABLE
\bTR \bTD 1.30 -- 2.15  \eTD \bTD Dr Ronika Power  \eTD \bTD The Unquiet Dead: Politics and Ethics of Studying Archaeological Human Remains \eTD \eTR
\bTR \bTD 2.15 -- 3  \eTD \bTD Dr Eve Guerry  \eTD \bTD Ancient Egyptian religion, temples and festivals in the Eighteenth Dynasty \eTD \eTR
\bTR \bTD 3 -- 3.30  \eTD \bTD Afternoon tea \eTD \eTR
\bTR \bTD 3.30 -- 4.15  \eTD \bTD A/Prof Shawn A. Ross  \eTD \bTD How Archaeology Informs History \eTD \eTR
\bTR \bTD 4.15 - 4.30  \eTD \bTD A/Prof Peter Keegan  \eTD \bTD Thank you \& close \eTD \eTR
\bTR \bTD 4.30 -   \eTD \bTD Tour of the exhibition  \eTD \bTD Escape from Pompeii: the Untold Roman Rescue \eTD \eTR
\eTABLE




{\bf Conference menu}

\startitemize
\item  Arrival

\startitemize
\item  Freshly brewed coffee and a selection of T2 teas
\item  Chilled water
\stopitemize

\item  Morning Tea

\startitemize
\item  Smoked salmon petit bagels with rocket \& cream cheese
\item  Cinnamon \& apple pies with clotted cream
\item  Gluten-free options
\item  Freshly brewed coffee and a selection of T2 teas
\item  Orange juice; chilled water
\stopitemize
\item  Lunch

\startitemize
\item  A selection of sandwiches, wraps \& artisan rolls
\item  Vegetarian \& gluten-free options
\item  Seasonal fruit platters
\item  Freshly brewed coffee and a selection of T2 teas
\item  Orange juice; chilled water
\stopitemize
\item  Afternoon Tea

\startitemize
\item  Glazed Danish pastries with fruit \& custard
\item  Freshly brewed coffee and a selection of T2 teas
\item  Orange juice; chilled water
\stopitemize
\stopitemize


\page

\tfx

\section{Dr Gil Davis} 

Dr Gil Davis is the Director, Program for Ancient Mediterranean Studies, at Macquarie University where he lectures in Greek history and runs the Ancient Israel Program. His primary research interests are numismatics, law and economics. He is the Managing Editor of the Journal of the Numismatic Association of Australia.

\subsection{Content} 

\startitemize
\item Option H -- Greece: The Greek World 500 -- 400 BC  

\startitemize
\item 2 -- Development of Athens and the Athenian Empire 
\stopitemize
\item Option G -- Pericles 

 \startitemize
 \item  4 -- Evaluation
 \stopitemize
  
\stopitemize


\subsection{Re-dating the Athenian Empire}

The last decade has seen a sea-change in the dating of Athenian inscriptions from the 5th century B.C. This is largely based on recent scholarly acceptance that the '3-bar sigma' letter-form should not be used as the determining chronological marker. The significance of some important epigraphic evidence must now be reinterpreted in changed contexts. This talk discusses the key evidence and the historiography of the debate. It considers whether the creation of the Athenian archē (empire) was intentional, and the extent to which imperialism should be linked with Perikles. 

\section{Dr Chris Forbes}

Dr Chris Forbes teaches Hellenistic history, Greek and Roman history of ideas and New Testament and early Christian history at Macquarie University. He is the Vice-President of the Society for the Study of Early Christianity at Macquarie, and his research focuses on the social, cultural and intellectual context of early Christianity.

\subsection{The Book of the Acts of the Apostles as a Source for the History of Earliest Christianity}

The earliest known Christian documents are the letters of St. Paul. Though they tell us occasional points of detail about Jesus, and contain occasional first-hand material about Paul's own past travels, current situation and future plans, they do not give us sufficient information to really begin writing a narrative history of even a small part of the early Christian movement. To tell the story of Jesus, and the story of the development of the movement which he founded, we are reliant on later, narrative sources. For Jesus himself we have primarily the four canonical Gospels. For the earliest Christian movement we have only the Acts of the Apostles, and a few scattered traditions in later first-century and early second-century Christian and secular writers.

The Acts of the Apostles is, then, a crucial source of information. Yet it is brief, highly selective, and its sources of information are largely unknowable. The lecture will discuss the approach an historian should take when attempting to write a history of earliest Christianity, given these factors.

\section{Dr Eve Guerry} 

Dr Eve Guerry is an archaeologist and teacher with expertise in the cultures of Ancient Egypt and Ancient Israel. She is the Roth Fellow for Ancient Israel School Outreach at Macquarie University and teaches Egyptian Archaeology. Eve completed her PhD in Egyptology at Macquarie University in 2010 on the topic of boundary transgression in Ancient Egyptian society and her areas of research interest centre around Ancient Egyptian religion and thought. Eve loves to teach and is fascinated by ancient concepts of our world and the divine. 

\subsection{Content} 

Ancient Egyptian Society in the New Kingdom until the death of Amenhotep III:


\startitemize[packed]
\item Religion
\item Temples (architecture, function)
\item Festivals
\stopitemize

\subsection{Ancient Egyptian religion, temples and festivals in the Eighteenth Dynasty.}

To what extent were ordinary people involved in the religion and cult of New Kingdom Egypt? We will examine evidence for the role of religion in everyday life during the 18th Dynasty and assess how and why it changed during this period.


\section{A/Prof Tom Hillard} 

Tom Hillard has been a member of the teaching staff of the Department of Ancient History at Macquarie University on and off since 1973. In the interim he has taught in the Department of Classics and Ancient History at the University of New England. His research focusses primarily on the politics of the Late Roman Republic, Roman social history and underwater archaeology. He has worked on archaeological sites in Greece, Syria and Israel, but Rome and Italy remain his first love.

\section{Dr Lea Beness} 

Lea Beness has been a staff member in the Department of Ancient History at Macquarie University since 1990. Her research focuses primarily on the politics of the Late Roman Republic. She is presently working on the Macquarie Dictionary of Roman Biography and a series of articles related to that project, as well as a long overdue monograph based on her doctoral dissertation: The Tribunate from Saturninus to Sulla, 103--88 BC. She is also writing up the results of geophysical survey work at Torone in the Chalkidike, Northern Greece, with her partner, Associate Professor Tom Hillard. She is the editor of Macquarie's journal Ancient History: Resources for Teachers and Convenor of the Department of Ancient History at Macquarie.

\subsection{The Late Roman Republic in 2017: Recent Developments}

In this session we shall survey recent developments in the study of the Roman Republic. In particular, we shall explore (briefly) the recent wave of demographic studies seeking to quantify the strength of the Roman State in terms of its 'human capital'; the current interest in the Roman Republic's political culture, the role of 'public opinion' and notions of the 'rule of law' (a report on four international symposia held in 2016 and 2017); the evolving nature of military commands; and the increasingly appreciated role of religio in public affairs and the concept of the predestined Leader (of obvious consequence to the decline of the Republic).
 
We hope to leave time for a Q\&A to follow.

\subsection{Syllabus topics addressed}


\startitemize[packed]
\item  Option L: Rome: Political revolution in Rome 133 -- 78 BC

\startitemize[n]
\item     1. Developments in Rome
\item     2. Wars and the impact of the empire
\stopitemize

\item  Option M: Rome: the fall of the Republic 78 -- 31 BC

\startitemize[n]
\item     1. Political developments in the late Republic
\item     2. Wars and expansion
\item     3. Fall of the Republic
\stopitemize

\item  Option J Rome: Tiberius Gracchus

\startitemize[n]
\item     1. Historical context
\item     3. Career
\stopitemize

\item  Option K Rome: Julius Caesar

\startitemize[n]
\item     1. Historical context
\item     3. Career
\item     4. Evaluation
\stopitemize

\stopitemize

\section{Ms Katherine Jacka} 

Katherine is in the last year of a PhD in the Department of Arabic Language and Cultures at the University of Sydney. Her research deals with Arabic scholarship produced in the Kingdom of Sicily in the twelfth century. Katherine has presented at a variety of international conferences in the UK and USA as well as in Australia and New Zealand. She has published in the journal Parergon and The Conversation.



\subsection{Persian Polymath: Ibn Sīna and the Development of Islamic Philosophy and Science}


Ibn Sīna (c. 980 -- 1037 CE), known as Avicenna in the Latin West, is one of the most distinguished intellectuals in the history of Islamic philosophy and science. Although a prolific writer in the Arabic language, Ibn Sīna hailed from modern day Uzbekistan and his native language was Persian. Educated at Bukhara, then one of the premier intellectual capitals of the Islamic world, Ibn Sīna was greatly influenced by the corpus of books found in the libraries of the Samanid princes which contained original works in Persian. 

While the influence of Greek scholarship on Islamic philosophy and science has been well-documented, research on the connection with original Persian work has received far less attention. Through the example of Ibn Sīna, this talk will explore the importance of Persian culture and scholarship on the wider field of Islamic philosophy and science in the period that has come to be known as Islam's 'Golden Age'.



\section{Dr Kyle H. Keimer} 

Dr Kyle Keimer is a Lecturer in the Archaeology and History of Ancient Israel and the Near East at Macquarie University. His research focuses on: the interface of ancient warfare, geopolitical, and socioeconomic concerns; and, on the political and social structure of ancient Israel in the early Iron Age. Presently he is co-editing a volume entitled Getting the Message Across: Communications in the Ancient Near East (Routledge 2017 forthcoming).

\subsection{The Early Israelite Monarchy in Text and Archaeology}

Reconstructing the early Israelite monarchy is fraught with difficulty. Textual sources such as 1 and 2 Samuel are complex literary productions that have undergone later editing and/or record partial historical details relevant for the theological narrative of the Bible. Often, only snapshots of historical events and processes are preserved. When further understanding of ancient Israelite society is sought via archaeology, there are an equal number of constraining issues, including the accurate dating of archaeological materials and the interpretation of archaeological remains, including the social reconstructions that result. Despite such textual and archaeological challenges, however, advancements in understanding ancient Israelite society and the early Israelite monarchy are being made. When both text and archaeology are interpreted together and viewed within a sociologically informed hermeneutic, a more lucid and profound understanding of the early Israelite monarchy is possible. This presentation will discuss some of the various issues and approaches to understanding the early Israelite monarchy of the biblical texts and the social reality behind those texts.



\section{Prof Ray Laurence}

Macquarie's new Professor of Roman History, Professor Ray Laurence, currently Professor of Roman History and Archaeology in the School of European Culture and Languages at the University of Kent, and author, inter alia, of Pompeii: The Living City (with A. Butterworth 2005), Roman Pompeii: Space and Society (2006), Roman Passions: A History of Pleasure in Imperial Rome (2010) and Roman Archaeology for Historians (2012)

\subsection{Keynote: Children in Pompeii -- the invisible majority?}

A gap in the study of Pompeii is the presence of children in significant numbers within the city. My keynote lecture will seek to consider how this can be remedied and also to consider whether children could participate in the religious life of the city.  How do we place children into the cities of Pompeii and Herculaneum?  My paper will draw on recent research on childhood to seek to incorporate the relationship between the agency of the child and the urban environment. I will seek to establish a framework for the study of the child in the city that will draw on recent developments within urban studies more generally. This analysis will feature a re-conceptualization of my own work on neighbourhoods of Pompeii to incorporate the world of the child. In addition, I will attempt to move forward to consider the extent to which Pompeii was a child-centered city.


Over 4.5 million people have viewed the film 'A Glimpse of Teenage Life in Ancient Rome' 

https://www.youtube.com/watch?v=juWYhMoDTN0


\section{Dr Ronika Power} 

Dr Ronika Power is a Lecturer in Bioarchaeology at Macquarie University, an Honorary Fellow of the McDonald Institute for Archaeological Research, University of Cambridge, a member of the Big History Institute and the Editor of the British Association for Biological Anthropology and Osteoarchaeology Annual Review. Her research methodology is Biocultural Archaeology, whereby data derived from scientific analyses of the human body is interpreted in conjunction with all other forms of archaeological and historical evidence to provide meaningful insights into the structure, health, life-ways and world-views of individuals and groups from past populations. Dr Power has worked with geographically and temporally diverse populations: from early Holocene hunter-gatherers of Kenya; to megalithic temple builders of Neolithic Malta; multi-period cemeteries across Egypt; the Garamantes of the Pre-Islamic Libyan Sahara; Amarna Period Egyptian colonies in Nubia; Late Anglo-Saxon English child, infant and foetal burials; settlement interments in Medieval Benin; and post-14th century palace burials from the Maldives, to name a few.    


\subsection{The Unquiet Dead: Politics and Ethics of Studying Archaeological Human Remains}


This talk will explore ways to approach the ethical issues related to the excavation, presentation, ownership and custodianship of 
archaeological human remains.



\section{Dr Louise M. Pryke} 

Dr Louise M. Pryke is the Lecturer for Languages and Literature of Ancient Israel at Macquarie University, and an Honorary Associate of the University of Sydney. Louise's research is focused on the myths, history and literature of the Ancient Near East. In 2016, Louise was the recipient of the International Association for Assyriology (IAA) Fund - an international award for promising early career scholars in the field of Assyriology. Her next book Ishtar will be published in 2017.

\subsection{Content} 

This talk is addressed at the BOS content of Sacred Texts (the Hebrew Bible) and Core ethical Teachings (The Book of Proverbs -- wisdom, righteousness, purity and generosity of spirit).


\subsection{The Book of Proverbs: Wisdom, Women, Worldliness}

The Book of Proverbs in the Hebrew Bible has been described by Alter as the 'anthology of anthologies.' This anthological quality of the book creates a contradictory and complex text, yet one that holds significance for understanding the moral and ethical universe underpinning the religion of Judaism. In this talk, we explore the Book of Proverbs, and consider it in the context of the broader genre of wisdom literature in the Ancient Near East.



\section{A/Prof Shawn A. Ross} 

Shawn A Ross (Ph.D. University of Washington, 2001) is Associate Professor of History and Archaeology and Deputy Director of the Big History Institute at Macquarie University in Sydney, Australia. A/Prof Rossʼs research interests include the history and archaeology of pre-Classical Greece, oral tradition as history (especially Homer and Hesiod), the archaeology of the Balkans (especially Thrace), Greece in its wider Mediterranean and Balkan context, and the application of information technology to research.


\subsection{How Archaeology Informs History}




Too often Classical Archaeology has sought to corroborate or 'prove' textual accounts using archaeology. The articulation of historical and archaeological evidence is, however, more complex, difficult, and potentially rewarding. Although textual and material evidence rarely speak to one another directly, each can contribute unique insights to an understanding of the past. Issues that are difficult to explore through one approach are often more amenable to the other. This talk will briefly review the history of archaeology, discuss how archaeological and historical approaches can be combined responsibly, and provide a case study from my own research.








\stoptext